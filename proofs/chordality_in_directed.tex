
\documentclass[12pt]{article}
 
\usepackage[margin=1in]{geometry} 
\usepackage{amsmath,amsthm,amssymb}
\usepackage{hyperref}

\newcommand{\N}{\mathbb{N}}
\newcommand{\Z}{\mathbb{Z}}
\newcommand{\C}{\mathbb{C}}
\newcommand{\R}{\mathbb{R}}
 \newcommand{\Q}{\mathbb{Q}}

\newtheorem{theorem}{Theorem}[section]
\newtheorem{lemma}[theorem]{Lemma}
\theoremstyle{remark}
\newtheorem*{remark}{Remark}
\theoremstyle{definition}
\newtheorem{definition}{Definition}[section]
\newtheorem{corollary}{Corollary}[theorem]

\begin{document}
 \title{Dichordality and Ditriangulation}
 \author{Merrick Chang}
 \maketitle

\section{Preliminaries}

\begin{definition} We call a directed edge, $e_{xy}$, a \textbf{dicord} if there exists some directed cycle, $C = (v_1,v_2, \dots, v_k, v_1)$, of length $k>3$ such that $1 \leq x < y \leq k$ and $v_x$ and $v_y$ are non-adjactent in the cycle. We say a directed graph is \textbf{dichordal} if for each directed cycle of length 4 or greater has a dicord.
\end{definition}

To meaningfully create a directed counterpart to triangulation algorithms, we need an analogue to the simplical vertices in undirected graphs. Unfortunately, the obvious analogue, that a vertex induces a clique, is denied. We thus pause on this point for a moment.

%\begin{definition}Let $G = (V,E)$ be a dicordal graph. Let $S_{C_3}(G)$ be the set of 3-cycles in G. Then, we call the graph $H = (T, F)$ the \textbf{triangular representation} of $G$ if it satisfies the following:
%\begin{enumerate}
%\item There exists a 3-cycle $C = (a,b,c,a)$ iff there exists a vertex $v_{abc} \in T$.
%\item For two 
%\end{enumerate}
%\end{definition}

\end{document}